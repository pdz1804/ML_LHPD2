\section{Future Developments}

Building upon the foundation established in this assignment, our future work—particularly in \textbf{Assignment 2}—will focus on advancing our sentiment analysis system through more sophisticated machine learning techniques. The key areas of improvement will include:

\begin{itemize}
    \item \textbf{Support Vector Machines (SVMs):} Implement kernel functions for text processing, optimize soft margin classification, and explore multi-class extensions for sentiment classification.
    \item \textbf{Dimension Reduction (PCA/LDA):} Apply feature selection techniques such as variance thresholding and topic modeling to handle high-dimensional sparse text data effectively.
    \item \textbf{Ensemble Methods:} Develop robust sentiment classifiers using bagging and boosting techniques, incorporating voting and model combination strategies to improve predictive performance.
    \item \textbf{Discriminative Models:} Implement feature-based linear classifiers, logistic regression, and conditional random fields (CRF) for sequence labeling to enhance sentiment sequence understanding.
    \item \textbf{Engineering Optimization:} Improve efficiency in handling large-scale text data by focusing on model scalability, memory-efficient implementation, and parameter optimization techniques.
    \item \textbf{Model Generalization and Performance Analysis:} Evaluate model robustness across different datasets, assess feature importance, and refine hyperparameter tuning methods.
\end{itemize}

By integrating these advanced techniques, our goal is to enhance the accuracy, efficiency, and adaptability of our sentiment analysis system. Through rigorous experimentation and optimization, we aim to develop a more reliable model that can generalize well across diverse textual datasets. This next phase will further solidify our expertise in sentiment analysis, bridging the gap between theory and real-world applications.
