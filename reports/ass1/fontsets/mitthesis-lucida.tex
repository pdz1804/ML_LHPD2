% From mitthesis package
% Documentation: https://ctan.org/pkg/mitthesis

\ProvidesFile{mitthesis-lucida.tex}[2024/10/13 v1.04 Load Lucida fonts]

%% Lucida Bright and Lucida Bright Math fonts (available from the TeX Users Group)

\ClassNoteNoLine{\ClassName}{Lucida Bright and Lucida Bright Math fonts}

\ifpdftex
	\RequirePackage[LY1]{fontenc}
	\RequirePackage[expert,vargreek,altbullet]{lucidabr}   % switch text and math fonts 
	% some other options: seriftt (much larger quad size!), lucidasmallscale % 2024/10/13
	\RequirePackage{bm}
\else
    \RequirePackage[warnings-off={mathtools-colon,mathtools-overbracket}]{unicode-math}
    % suppress tiresome warnings about lack of integration between mathtools and unicode-math
    % unicode-math loads the fontspec package

    % FEATURE 	DESCRIPTION 											DEFAULT
    % dlig 		discretionary ligatures (f-ligatures in typewriter fnt) off
    % liga 		standard ligatures (f-ligatures, not monospaced fonts) 	on
    % onum 		old style numbers (Lucida Bright only)					off
    % smcp 		small caps 	(Lucida Bright, upright only)				off
    % zero 		slashed zero (sans and monospaced fonts)				off
    % locl 		script/language specific (localized) forms 				on
    %
    % FEATURE 	FONT 						DESCRIPTION
    % ss01 		Lucida Sans & Typewriter 	use dotted zero: 𝟶 vs. default 0.
    % ss01 		use Lucida-standard shapes:
    % 			Lucida Grande Mono 			ØOQ vs. default 0OQ;
    % 			Lucida Console 				ØOQ vs. default 0OQ.
    
    \setmainfont{LucidaBrightOT}[% main rm
          Ligatures=TeX,
          Extension=.otf,
          UprightFont=*,
          ItalicFont=*-Italic,
          BoldFont=*-Demi,
    	  BoldItalicFont=*-DemiItalic,
    %     BoldItalicFeatures={Color=Purple4}, % fixed 2024/10/13
    %     UprightFeatures={Color=Salmon3},
    	  Scale=0.93,% increased from 0.92, 2024/10/13
    %	  Numbers ={OldStyle}, 
      ]
    %
    \setsansfont{LucidaSansOT}[% main sans
          Ligatures=TeX,
          Extension=.otf,
          UprightFont=*,
          ItalicFont=*-Italic,
          BoldFont=*-Demi,
    	  BoldItalicFont=*-DemiItalic,
    	  Scale=MatchUppercase,
      ]
    %
    \setmonofont{LucidaSansTypewriterOT}[% main typewriter
          Ligatures=TeX,
          Extension=.otf,
          UprightFont=*,
          ItalicFont=*-Oblique,
          BoldFont=*-Bold,
          BoldItalicFont=*-BoldOblique,
          Scale=MatchLowercase,% changed from MatchUppercase, 2024/10/13
      ]
    
    %% MATH FONTS
    %
    % ss01 	alternate Latin lower case math italic identical to text italic. This option is not recommended as Greek 𝜈 (nu) and Latin 𝑣 (vee) become too similar
    % ss02 	smaller arrows (original Type 1 size): ⇑ vs. default ⇑.
    % ss03 	attempt to imitate Type 1 fonts
    % ss04 	calligraphic alternative (similar to the Type 1 design) to the math script alphabet
    % ss05 	use a slashed-zero style for the empty set character (U+2205): Ø vs. default ∅
    % ss08 	emphasize right and bottom edges of the d’Alembertian operator (U+29E0, \laplac), to follow the design of the Laplace operator (U+2206, \increment, Δ); the default design emphasizes the left and upper edges. Also available in the bold math font.
    %
    \setmathfont{LucidaBrightMathOT}[% main math
    	  Scale=MatchUppercase,
    	  Extension=.otf,
    	  BoldFont=LucidaBrightMathOT-Demi.otf,
%    	  BoldFeatures={Color=Chartreuse4},
    	  RawFeature ={+ss04,+ss08}, %
    ]
 
%% LucidaMath-OT contains a *serif* typerwriter font. If you prefer a sans serif typerwriter font, you can use the following two settings. 
% 
%    \setmathfont{LucidaSansTypewriterOT}[% <== NB: setting a text font where math font should go.
%    	  Scale=MatchUppercase,
%          range=tt,
%    	  Color=NavyBlue,
%          Extension=.otf,
%          script-features = {},     
%          sscript-features = {},     
%    	  BoldFont=LucidaSansTypewriterOT-Bold.otf,
%    	  BoldFeatures={Color=Salmon3},
%    ]
%    %
%    \setmathfontface\mathtt{LucidaSansTypewriterOT-Bold}[% bold upright typewriter font in boldmath.
%    	  version=bold,
%    	  Scale=MatchUppercase,
%          Extension=.otf,    
%    ]
	%
    \setmathfont{LucidaBrightMathOT}[%
    	  range={scr},% to make scr differ from cal
    	  RawFeature={-ss04},
    	  Scale=MatchUppercase,
          Extension=.otf,
    	  BoldFont=LucidaBrightMathOT-Demi.otf,% could be overridden by bold separate version
    ]
    \setmathfont{LucidaBrightMathOT}[% -Demi .. sufficient this way, since bold set above.
    	  range={bfscr},% need to call \symbfscr manually - bug in unicode-math version & range interaction
    	  RawFeature={-ss04},
    	  Scale=MatchUppercase,
          Extension=.otf,
    	  BoldFont=LucidaBrightMathOT-Demi.otf,
    ]
    %% Fix unicode-math operator font not set up for bold face (default \mathrm). See https://github.com/wspr/unicode-math/issues/468
    \setoperatorfont\normalfont
    %
    % note \mathbfit = \symbfit
    
    % The specialized one-off Lucida fonts:
    \newfontface\LucidaBlackletter{LucidaBlackletterOT}[%
    	Scale=0.95,% <== NB
    	Extension=.otf,
    	]
    \newfontface\LucidaCalligraphy{LucidaCalligraphyOT}[%
    	Scale=MatchUppercase,
    	Extension=.otf,
    	]
    \newfontface\LucidaHandwriting{LucidaHandwritingOT}[%
    	Scale=MatchUppercase,
    	Extension=.otf,
     	]
    % GrandeMono and Console fonts:
    \newfontface\LucidaGrandeMonoDK{LucidaGrandeMonoDK}[%	
    	Scale=MatchUppercase,
    	Extension=.otf,
        UprightFont=*,
        ItalicFont=*-Italic,
        BoldFont=*-Bold,
        BoldItalicFont=*-BoldItalic,
    	]
    \newfontface\LucidaConsole{LucidaConsoleDK}[%
    	Scale=MatchLowercase,% <=== NB
    	Extension=.otf,
        UprightFont=*,
        ItalicFont=*-Italic,
        BoldFont=*-Bold,
        BoldItalicFont=*-BoldItalic,
    	]
\fi
