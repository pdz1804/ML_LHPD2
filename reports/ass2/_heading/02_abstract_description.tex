\chapter{Abstract}

With the increasing adoption of Natural Language Processing (NLP) in various domains, sentiment analysis has become a crucial task in understanding opinions, emotions, and attitudes expressed in text. The ability to automatically classify sentiments in text is highly valuable for applications in social media monitoring, product reviews, and customer feedback analysis. This project aims to develop a Machine Learning (ML) model capable of performing sentiment analysis, distinguishing between different sentiment classes such as positive, negative, and neutral. Our goal is to explore various techniques in sentiment classification, evaluate model effectiveness, and contribute to advancements in automated sentiment analysis. 

\chapter{Description}

In the digital age, people share emotions and opinions through social media, product reviews, and forums. Sentiment analysis, or opinion mining, is a crucial NLP technique for businesses, researchers, and policymakers to analyze public sentiment. However, challenges like sarcasm, ambiguity, and varied linguistic expressions make accurate classification difficult.\\

For this project, our team (\textbf{LHPD2}) will develop a Machine Learning model for sentiment analysis as part of \textbf{Assignment 2} in this \textbf{Machine Learning} course. Our goal is to classify text into sentiment categories using NLP techniques like word embeddings, recurrent neural networks, and transformer-based models. This involves feature extraction, evaluating classification algorithms, and analyzing model performance.\\

Sentiment analysis plays a growing role in applications such as customer experience improvement and social media trend detection. However, ethical concerns, including bias in training data and misinterpretation of sentiments, must be addressed. Alongside implementing sentiment classification models, we will explore ways to enhance model fairness and accuracy.\\

This assignment focuses on applying \textbf{engineering techniques} to sentiment analysis using MODELS. Key tasks include \textbf{feature transformation, handling high-dimensional data, network architecture design, hyperparameter tuning, and model evaluation}. Additionally, we will apply \textbf{feature selection, optimization strategies, and probability modeling} to improve performance, aligning with \textbf{data preprocessing, model tuning, and performance analysis}.\\

Our objective is to gain hands-on experience by focusing on \textbf{structured implementation, tuning, and evaluation}, rather than theoretical innovations. This project will emphasize \textbf{efficient engineering solutions} to improve sentiment classification across multiple models.

\newpage
